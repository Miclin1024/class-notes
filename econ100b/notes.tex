\documentclass[a4paper]{article}

%% Language and font encodings
\usepackage[english]{babel}
\usepackage[utf8x]{inputenc}
\usepackage[T1]{fontenc}

%% Sets page size and margins
\usepackage[a4paper,top=3cm,bottom=2cm,left=3cm,right=3cm,marginparwidth=1.75cm]{geometry}

%% Useful packages
\usepackage[export]{adjustbox}
\usepackage{amsmath}
\usepackage{graphicx}
\usepackage[colorinlistoftodos]{todonotes}
\usepackage[colorlinks=true, allcolors=blue]{hyperref}
\usepackage{float}
\usepackage{enumerate}
\usepackage{subfig}
\usepackage{sectsty}
\graphicspath{ {./images/} }
%\subsubsectionfont{\underline}

\title{Economic Analysis Macro - Class Notes}
\author{Michael Lin \\ Professor: R. J. Hawkins}

\begin{document}
\maketitle

\section{Measurement}

We focus on the major items of macroeconomic interest: \textbf{Economic Activity, Inflation, Unemployment, Interest Rates}

\subsection{National Income Accounting}
\subsubsection*{The Three Approaches to National Income Accounting}
\underline{\textbf{Product Approach}}: the dollar amount of output \textbf{produced}. \smallskip \\
The current \textbf{market value} of all \textbf{final goods and services newly produced} in the domestic economy, 
during a specified period of time.
\begin{enumerate}
    \item \textbf{Market Value} \begin{itemize}
        \item Not everything has a market - \textbf{imputed values} must be used. Determined based on \textbf{production cost}. 
        \underline{Example: Fire Department.}
        \item Most non-market goods and services are not included. \underline{Example: Household services.}
        \item Some market goods and services are not included. \underline{Example: Used goods.}
    \end{itemize}
    \item \textbf{Final Goods and Services} \smallskip \\
    Goods and services that are \textbf{not completely used up} in the production process.
\end{enumerate}
\underline{\textbf{Expenditure Approach}}: the dollar amount \textbf{spent} by purchasers. \smallskip \\
The \textbf{total spending} on \textbf{all final goods and services} produced in the domestic economy during a specified period of time.
\begin{equation}
    \tag{\textbf{The National Income Identity}}
    {\boxed{Y = C + I + G + NX}}
    \end{equation}
    \begin{enumerate}
        \item \textbf{Consumption}\\
        Spending by domestic households on final goods and services.
        \item \textbf{Investment}\\
        Spending for new capital goods\footnote{\textbf{Capital goods} are used to produce other goods and services.
        } (fixed investment) plus inventory investment.
        \item \textbf{Gov. Purchases of Goods and Services} \\
        \textbf{Government Purchases} of goods and services. \smallskip \\
        \textbf{Government Transfer Payments} are \underline{\textbf{not}} included in $G$. (Not payments for goods and services) 
        \underline{Example: Medicare, Medicaid, Veterans’ benefits and etc.}
        \item \textbf{Net Exports}: Exports - Imports
    \end{enumerate}
\pagebreak
\textbf{\underline{Income Approach}}: the dollar \textbf{incomes earned} by production. \smallskip \\
The \textbf{total income} earned by individuals and businesses in the economy. \smallskip \\
\subsubsection*{The Five Income Measures}
\begin{itemize}
    \item \textbf{National Income} = Compensation of Employees + Other Income + Corporate Profits
    \item \textbf{Gross National Product (GNP)} = National Income + Depreciation
    \item \textbf{Gross Domestic Product (GDP)} = GNP + Net Factor Payments
    \item \textbf{Private Disposable Income (PDI)} = GDP + Net Factor Income + Transfer Payments from the Government 
    + Interest Payments on Government Debt - Taxes
    \item \textbf{Net Government Income (NGI)} = Taxes - Government Transfer Payments - Interest Payments on Government Debt
\end{itemize}

\subsection{Inflation}
\subsubsection*{Price Index}
Measures the average level of prices for some specified set of goods and services 
\textbf{relative to the prices in a specified base year.}
\subsubsection*{Three Major Price Indices} \begin{itemize}
    \item \textbf{The Gross Domestic Product (GDP) Deflator:}
    \[ P = 100 * \frac{\text{nominal GDP}}{\text{real GDP}} \]
    \item \textbf{The Personal Consumption Expenditure (PCE) Deflator:}
    \[ P = 100 * \frac{\text{nominal PCE}}{\text{real PCE}} \]
    \item \textbf{The Consumer Price Index (CPI):} \smallskip \\
    Measures the average prices of a specified basket of goods and services bought by consumers.
\end{itemize}
\subsubsection*{Measuring Inflation}
Given the price index at time $t$, denoted as $P_t$, the inflation rate $\pi_t$ is expressed as the following
\[ \pi_t \approx \frac{P_t - P_{t-1}}{P_{t-1}} \]
Where with a little bit algebra, we derive the \textbf{Fisher Equation}
\[ i \approx r + \pi \]
$i$ stands for the nominal interest rate, and $r$ is the real interest rate.
\subsubsection*{Unemployment}

\section{Aggregate Production and Productivity}
When labor and capital are separately increased with the other held constant, 
the product increases by \textbf{diminishing increments}.
\subsection{The Cobb-Douglas Production Function}
Their assumptions for $Y(K, L)$: \begin{itemize}
    \item Output $Y$ is a function of capital $K$ and labor $L$
    \item Constant returns to scale: \smallskip \\
    If $K - m * K$ and $L - m * L$ then $Y - m * Y$
\end{itemize}
Their proposed solution:\begin{equation}
    \tag{\textbf{The Cobb-Douglas Production Function}}
    \boxed{Y = AK^\alpha L^{1 - \alpha}}
\end{equation}
Shows how much output can be produced from givenamounts of 
capital and labor with a given level of total-factor productivity $A$ 
(also referred to as "technology"). \smallskip \\
Taking logs of both sides and subtract $\ln(L)$
\[\ln(\frac{Y}{L}) = \ln{A} + \alpha \ln(\frac{K}{L})\]
Notice that we now have a linear equation that we can analyze
\begin{figure}[H]%
    \centering
    \subfloat[\centering Linear Approx]{{\includegraphics[width=6cm]{agg_production_linear.png} }}%
    \qquad
    \subfloat[\centering Diminishing Marginal Return to K]{{\includegraphics[width=7cm]{Screen Shot 2020-09-18 at 4.12.59 PM.png} }}%
    \caption{}%
    \label{fig:1}%
\end{figure}

The production function can be drawn as either \begin{itemize}
    \item Output as a function of capital (As shown in Figure 1(b))
    \item Output as a function of labor
\end{itemize}
\begin{equation}
    \tag{Marginal Product of Capital}
    \begin{split}
        \text{MPK} & \equiv \frac{\partial Y}{\partial K} \\
        & = \alpha A (\frac{L}{K})^{1 - \alpha}
    \end{split}
\end{equation}

\subsection{Understanding Shocks}
Consider the Cobb-Douglas Production Function
\[ Y = AK^\alpha L^{1 - \alpha} \]
When $AL^{1 - \alpha}$ goes down, output goes down across all input $K$
\begin{figure}[H]%
    \centering
    \subfloat[\centering Reduced Y Across K]{{\includegraphics[width=6cm]{Screen Shot 2020-09-18 at 4.48.01 PM.png} }}%
    \qquad
    \subfloat[\centering Reduced MPC]{{\includegraphics[width=6cm]{Screen Shot 2020-09-18 at 4.48.27 PM.png} }}%
    \caption{Shock in $AL^{1 - \alpha}$}%
    \label{fig:Shock}%
\end{figure} \ \\
Moveover, we can see that the slope of the post-shock production curve is smaller 
than the pre-shock curve, which indicates the decrease of \textit{MPK}. As shown in 
Figure 2(b), this also indicates the decrease of \textbf{real rental cost of capital}. \smallskip \\
Similarily, when $AK^{\alpha}$ goes down, we will have a lower output across all input of capital.

\subsection{Full-Employment Output}
Full-employment, or potential, outputis the level of outputwhen the labor market is in the \textbf{long-run equilibrium}.
\[ Y^P = AK^\alpha L_S^{1-\alpha} \]

\section{The Solow-Swan Model: Long-Run Growth}
Over the decades, we discovered that the growth of GDP is roughly \textbf{linear} in the \textbf{logrithmic domain}. 
\[ \ln(Y(t)) = c + g_Y t \]
Following this idea, we can get a model for describing the output at time $t$
\[ Y(t) = Y(0)e^{g_Y t} \]
\includegraphics[width=0.5\textwidth, center]{Screen Shot 2020-09-25 at 5.36.52 PM.png}
We can see that our log-linear fit is pretty reasonable. The fluctuations about the trend are referred to as \textbf{bussiness cycles}.

\subsection{Where Does Growth Come From?}
Let's begin with the Cobb-Douglas Production Function
\[ Y = AK^\alpha L^{1 - \alpha} \]
Taking the log of both sides, then take the derivative
\begin{equation}
    \tag{Growth Accounting Formula}
    \boxed{\frac{1}{Y}\frac{dY}{dt} = \frac{1}{A}\frac{dA}{dt} + \alpha \frac{1}{K}\frac{dK}{dt} + (1-\alpha)\frac{1}{L}\frac{dL}{dt}}
\end{equation}
Approximate the result with $\Delta t = 1$, we have
\[ \frac{\Delta Y}{Y} = \frac{\Delta A}{A} + \alpha \frac{\Delta K}{K} + (1 - \alpha)\frac{\Delta L}{L} \]
Sometimes it's also written as $g_Y = g_A + \alpha g_K + (1 - \alpha)g_L$ \smallskip \\
For example, in the United States where $\alpha \approx 0.3$, we have
\[g_Y = g_A + 0.3 g_K + 0.7 g_L\]
This tells us how the growth of one or more of the variables in the production function attribute to the overall increase in output.\smallskip \\
Research has shown that \textbf{productivity growth} is a more important source of variation in growth rates across countries than is growth 
in capital or labor(a.k.a. factor accumulation).

\subsection{The Solow-Swan Model}
The Solow-Swan Model was developed in 1950s to determine \textbf{capital accumulation}, which affects \textbf{economic growth}. \smallskip \\
To derive this, we assume that a \textbf{constant fraction $s$ of output} $Y$ is saved, and the capital \textbf{depreciates at a constant rate} $\delta$.
\begin{equation}
    \tag{Capital Accumulation Equation}
    \frac{dK}{dt} = sAK^{\alpha}L^{1-\alpha} - \delta K
\end{equation}
The goal is to solve this differential equation so we get $K$ as a function of $t$ \smallskip \\
We make the following assumptions: \begin{itemize}
    \item Replace total factor of productivity $A$ with \textbf{labor efficiency} $E$, and assume that $E(t)$ has a \textbf{log-linear growth rate}
    \begin{align*}
        E(t) &= A(t)^{1/(1-\alpha)} \\
        E(t) &= E(0)e^{g_E t}
    \end{align*}
    \item Assume that $L$ also grows in log-linear rate.
    \[ L(t) = L(0)e^{g_L t} \]
    \item Instead of examine the aggregate capital $K$, we look at the the \textbf{capital per worker per unit of worker efficiency}, denoted by $\kappa$. This 
    is also referred to as \textbf{normalized capital}.
    \[ \kappa(t) = \frac{K(t)}{E(t)L(t)} \]
\end{itemize}
Plug in our expression for $\kappa(t)$, $L(t)$, and $E(t)$, we have the capital accumulation equation for $\kappa$
\[ \frac{d\kappa}{dt} = s\kappa^\alpha - (g_E + g_L + \delta)\kappa \]
The solution $\kappa(t)$ is important because it is directly related to \textbf{per-capita income} by
\[ \frac{Y(t)}{L(t)} = \kappa(t)^\alpha E(t) \]
% 
Generally we have two solutions of interest
\begin{itemize}
    \item The steady-state solution $\kappa^*$
    \[ \frac{d\kappa}{dt} = 0 \]
    \item The general solution $\kappa(t)$
\end{itemize}

\pagebreak

\subsection{The Solutions}

\subsubsection*{Stead-state Solution}
Follows from the capital accumulation equation
\[ \frac{d\kappa}{dt} = s\kappa^\alpha - (g_E + g_L + \delta)\kappa = 0 \]
so \begin{align*}
    s\kappa^\alpha &= (g_E + g_L + \delta)\kappa \\
    \kappa^{1-\alpha} &= \frac{s}{g_E + g_L + \delta}
\end{align*}
and
\begin{equation*}
    \boxed{\kappa^* = (\frac{s}{g_E + g_L + \delta})^\frac{1}{1 - \alpha}}
\end{equation*}
In the steady state, per-capita income is given by
\[ \frac{Y(t)}{L(t)}  = \kappa^*(t)^\alpha E(t) = (\frac{s}{g_E + g_L + \delta})^\frac{1}{1 - \alpha}E(t) \]
and from which follows that
\[ g_{Y/L} = g_E \]
In the steady state $\kappa(t) = \kappa^*$ and is not changing, so
\[ \kappa(t) = \frac{K(0)e^{g_Kt}}{E(0)e^{g_Kt}L(0)e^{g_Kt}} 
= \frac{K(0)}{E(0)L(0)}e^{(g_K-g_E-g_L)t} = \kappa^* \]
Therefore $g_{Y/L} = g_E = g_K - g_L$

\subsubsection*{General Solution}
The general solution can be written in 'gap' form
\[ \text{gap}(t) = \text{gap}(0)e^{-t/\tau} \]
where
\[ \text{gap}(t) = \kappa(t)^{1-\alpha} - \frac{s}{g_E + g_L + \delta} \]
Overtime $\text{gap}(t)$ converges to 0, with the \textbf{half life} given by
\[ \boxed{t_{1/2} = \frac{\ln 2}{(1 - \alpha)(g_E + g_L + \delta)}} \]

\subsection{Shock Analysis}
Four important components\begin{itemize}
    \item $\kappa^*(t)^\alpha E(t)$: \textbf{The balanced growth} 
    \[\kappa^* = (\frac{s}{g_E + g_L + \delta})^\frac{1}{1 - \alpha}\]
    \item $Y(t)/L(t)$: \textbf{Per capita income}
    \[ \frac{Y(t)}{L(t)} = \kappa(t)^\alpha E(t) = (\frac{K(t)}{E(t)L(t)})^\alpha E(t) \]
    \item $t_{1/2}$: \textbf{Half life} -- How fast per capita income converges to the 
    balanced growth after a shock
    \[ t_{1/2} = \frac{\ln 2}{(1 - \alpha)(g_E + g_L + \delta)} \]
    \item $g_{Y/L}$: \textbf{Long term growth rate}
    \[ g_{Y/L} = g_E \]
\end{itemize}

\subsection{Endogenous Growth Theory - the Romer Model}
If we are looking for the factors that help the economy grows faster, we soon 
realize that factors such as the saving rate, labor-force growth rate can rise 
and fall forever, but efficiency can always improve. \smallskip \\
In endogenous growth theory, we look at \textbf{technology as a 
production input.} It differs from capital and labor in two important 
characteristics: \begin{itemize}
    \item \textbf{Non-rivalry: } More than one person can use the factor 
    at any given time.
    \item \textbf{Non-excludability: } One person cannot prevento thers 
    from using the factor.
\end{itemize}
Endogenous growth theory is an attempt to explain how 
and why technology can increase
endogenously and, thereby sustained increases in income-per-worker. 
It is often referred to as the \textbf{Romer Model}.

\subsubsection*{The Romer Model}
In the Romer Model, \textbf{labor} is allocated to the production of \begin{itemize}
    \item \textbf{Goods and services} $L_P$
    \item \textbf{New technology} $L_E$
\end{itemize}
The total labor supply $L  = L_P + L_E$ is assumed to be fixed. Let $\gamma$ 
denotes the ratio of total labor participating in R\&D.
\[ L_E = \gamma L \]
The production function for technology
\[ \frac{dE}{dt} = \chi E L_E \]
where $\chi$ indicates how productive labor is in producing ideas. \smallskip \\
%
If we take the technology production function and divide both side by $E$
\[ \frac{1}{E}\frac{dE}{dt} \equiv g_E = \chi L_E \]
Substituting $L_E$, we get the \textbf{Romer Model}
\[ \boxed{g_E = \chi \gamma L} \]
%
\section{Bussiness Cycles}
Business cycles are \textbf{short-run fluctuations} in aggregate economic activity 
around its long-run growth path. Many economic activities will expand and contract 
together in a recurring \textbf{but not periodic} fashion. \smallskip \\
%
Macroeconomic variables exhibit co-movement with GDP. They have regular and
predictable patterns over the course of business cycle. They can be 
classified by the \textbf{direction, timing, and volatility.} \smallskip \\
%
The \underline{\textbf{direction}} of the co-movement can be: \textbf{procyclical, countercyclical, or
acyclical.} 
\smallskip \\
The \underline{\textbf{timing}} of the co-movement can be: 
\textbf{leading, coincident, lagging, or
not designated.} \smallskip \\
The \underline{\textbf{volatility}} of the co-movement can be: \textbf{higher, similar, lower, or
not designated.}  


\section{IS Curve}
From the production function we can derive the economy's 
\textbf{full-employment output level} 
(how much the economy could produce). This is the basis for the 
economy's \textbf{aggregate supply.} \smallskip \\
%
Now the focus shifts to how much is the demand in the economy, a.k.a the 
\textbf{aggregate demand}. \smallskip \\
%
Aggregate demand and aggregate supply are crucial for explaining \textbf{short-run 
fluctuations} in economic activity.
\subsection{Planned Expenditures}
There are two types of expenditure \begin{itemize}
    \item \textbf{Acutal Expenditures} $Y$: Total amount of spending on 
    domestically produced goods and services that households, businesses, the 
    government, and foreigners acutally make. \textbf{This is equivalent to the \underline{total 
    output} actually produced in the economy.}
    \item \textbf{Planned Expenditures} $T^{\text{pl}}$: Total amount of spending on 
    domestically produced goods and serves that households, businesses, the 
    government, and foreigners want to make. \textbf{This is quivalent to the 
    \underline{aggregate demand}.}
\end{itemize}
The economy is in \textbf{goods market equilibrium} when actual expenditures equal planned 
expenditures
\[ Y = Y^{\text{pl}} \]
Similarily to the national income identity, $Y^{pl}$ can be expressed as
\[ Y^{\text{pl}} = C + I^{\text{pl}} + G + NX \]

\subsubsection*{The Consumption Function}
Consumption expenditures are represented by the \textbf{consumption function}
\[ C = \overline{C} + \text{mpc} \cdot Y^d - \zeta_c  r_L \]
where
\begin{align*}
    \overline{C} &\equiv \text{Autonomous Consumption Expenditure} \\
    Y^d &\equiv \text{Disposable Income} = Y - T \\
    \text{mpc} &\equiv \text{Marginal Propensity to Consume} \\
    \zeta_c &\equiv \text{Sensitivity of $C$ to $r_L$} \\
    r_L &\equiv \text{Real Lending Interest Rate}
\end{align*}
\textbf{Autonomous consumption } $\overline{C}$ includes those factors that
affect consumption expenditures but are not explicitly included in the model.
\smallskip \\
This would include \begin{itemize}
    \item Consumer confidence or sentiment (+)
    \item Household wealth (+)
    \item Expected future income (+)
\end{itemize}

\subsubsection*{The Investment Function}
Planned investment expenditures are represented by the 
\textbf{investment function}
\[ I^{\text{pl}} = \overline{I} - \zeta_I \times r_L \]
Similar to $\overline{C}$, $\overline{I}$ is affected by \begin{itemize}
    \item Business confidence or sentiment (+)
    \item Expected future profit and/or cash flow (+)
    \item Changes in technology (+)
\end{itemize}

\subsubsection*{Net Exports Function}
Net exports has 2 components: Exports and Imports. It is represented by the
\textbf{net export function}
\[ \text{NX} = \overline{\text{NX}} - \zeta_{\text{NX}} \times r_L \]
Autonomous net exports can be affected by \begin{itemize}
    \item Domestic preferences for foreign goods (-)
    \item Foreign preferences for domestic goods (+)
    \item Foreign trade barriers (-)
\end{itemize}

\subsubsection*{Government Spending}
Government purchases are mainly autonomous $G = \overline{G}$, as well as
taxes $T = \overline{T}$. They are both determined within the political system.

\subsection{Goods Market Equilibrium}
Summerizing our equations so far. At goods market equilibrium 
$Y = Y^{\text{pl}}$, we have
\[ Y = \frac{\overline{C} + \overline{I} + \overline{G} + \overline{\text{NX}}
- \text{mpc} \times \overline{T}}{1 - \text{mpc}} - \frac{\zeta_C +
 \zeta_I + \zeta_{\text{NX}}}{1 - \text{mpc}}r_L \]
The \textbf{lending rate} $r_L$ is the interest rate at which companies 
can borrow money
for. It is set by the commercial banking system. It has the following 
relationship with the \textbf{policy rate} $r$, which is set by the central bank.
\[ r_L = r + s_{\text{term}} + s_{\text{credit}} \]
\includegraphics[width=.5\textwidth, center]{Screen Shot 2020-10-15 at 12.56.31 AM.png}
Substituting $r_L$, the IS curve becomes
\[ Y = \frac{\overline{C} + \overline{I} + \overline{G} + \overline{\text{NX}}
- \text{mpc} \times \overline{T} - \overline{S}}{1 - \text{mpc}} - \frac{\zeta_C +
\zeta_I + \zeta_{\text{NX}}}{1 - \text{mpc}}r \]
where $(\overline{S} = \zeta_C + \zeta_I + \zeta_{\text{NX}})
(s_{\text{term}} + s_{\text{credit}})$
\smallskip \\
from which we obtain a simplified expression for the IS curve
\[ \boxed{Y = \overline{Y} - \zeta_Y r} \]
if follows that there is a rate $r^*$ that produces potential output $Y^P$
\[ Y^P = \overline{Y} - \zeta_Y r^* \]

\section{The Phillips Curve}
The Phillips Curve is the \textbf{inverse relationship between inflation
and unemployment.} \smallskip \\
The intuition is that a lower unemployment rate means more people are working,
which signals increased demand of labor. This can drive up wages.
Because higher wages lead to a higher total cost of production, this 
translates into higher price and higher inflation. Secondly, when people 
have more disposable income they will be willing to buy things 
at a higher price. \smallskip \\
The Phillips Curve also implies an \textbf{inverse relationship 
between nominal wages and unemployment.}

\subsection{Natural Rate of Unemployment}
In the long-run, when all wages and prices are completely flexible, unemployment
would be at the natural rate of unemployment $u_N$. \smallskip \\
There are two components of $u_N$: \textbf{frictional unemployment and structural
unemployment}. \begin{itemize}
    \item Frictional unemployment involves people transitioning between jobs; 
    it has nothing to do with the economic cycle and is voluntary.
    \item Structural unemployment is seen as the change in the nature of the 
    economy that renders the skillsets associated with certain individuals no 
    longer optimal in that economy. It is a direct result of shifts in the 
    economy, including changes in technology or declines in an industry.
    \item Structural unemployment is very concerning to economists, while 
    frictional unemployment is considered inevitable and not factored into 
    the unemployment rate.
\end{itemize}
%
Incorporating these changes, we have the \textbf{expectations-augmented 
Phillips Curve}
\[ \pi = \pi^e - \omega(u - u_N) \]
where $\pi^e$ is the expected inflation. This implies that in the long-run
$u = u_N$, and therefore $\pi = \pi^e$. So there is 
\textbf{no long-run trade-off between 
unemployment and inflation. } \smallskip \\
Thus there are two types of Phillips Curve \begin{itemize}
    \item Short-run Phillips Curve (SRPC) \smallskip \\
    It shows the inverse relationship between inflation and unemployment.
    \item Long-run Phillips Curve (NAIRU) \smallskip \\
    It is a vertical line at the natural rate of unemployment.
\end{itemize}

\subsection{The Short-run Phillips Curve}
When we include price shocks $\rho$ we have the modern short-run Phillips
Curve
\[ \pi = \pi^e - \omega(u - u_N) + \rho \]
This implies that wages and prices are \textbf{sticky}.
\smallskip \\
The more flexible wages and prices are, the more they will respond to the
unemployment gap, and the more inflation will respond to the gap. i.e. 
$\omega$ is larger.
\smallskip \\
When wages and prices were completely flexible, $\omega = \infty$,
the SRPC becomes vertical and identical to the NAIRU.
\subsubsection*{An Example
\footnote{The Relationship Between Inflation and Unemployment. 
\textit{lumenlearning.com}}
}
\includegraphics[width=.4\textwidth, center]
{Screen Shot 2020-10-23 at 2.31.14 PM.png}
Assume the economy starts at point A and has an initial rate of unemployment 
and inflation rate. If the government decides to pursue expansionary economic 
policies, inflation will increase as aggregate demand shifts to the right. 
This is shown as a \textbf{movement along the short-run Phillips curve}, to point B, 
which is an unstable equilibrium. 
\smallskip \\
As aggregate demand increases, more workers 
will be hired by firms in order to produce more output to meet rising demand, 
and unemployment will decrease. However, due to the higher inflation, workers’ 
\textbf{expectations of future inflation changes}, which shifts 
the short-run Phillips 
curve to the right, from unstable equilibrium point B to the stable equilibrium 
point C. At point C, the rate of unemployment has increased back to its natural 
rate, but inflation remains higher than its initial level.
\smallskip \\
The reason the short-run Phillips curve shifts is due to the changes in 
inflation expectations. Workers, who are assumed to be completely rational 
and informed, will recognize their nominal wages have not kept pace with 
inflation increases (the movement from A to B), so their real wages have 
been decreased. As such, in the future, they will renegotiate their nominal 
wages to reflect the higher expected inflation rate, in order to keep their 
real wages the same. As nominal wages increase, production costs for the 
supplier increase, which diminishes profits. As profits decline, suppliers 
will decrease output and employ fewer workers (the movement from B to C). 
Consequently, an attempt to decrease unemployment at the cost of higher 
inflation in the short run led to higher inflation and no change in unemployment 
in the long run.

\subsection{Two Theories of Expectations}
There are two theories of expectations (adaptive or rational) that predict 
how people will react to inflation.
\smallskip \\
\textbf{Rational expectations theory} says that people use all information to
accuratly anticipate inflation. If the current inflation is higher than nominal
people will take that into consideration.
\smallskip \\
\textbf{Adaptive expectations theory} says that people use 
\textbf{past information} 
as the best predictor of future events. If inflation was higher than normal 
in the past, people will expect it to be higher than anticipated in the future.
Namely, this is saying that $\pi^e = \pi_{t-1}$
\bigskip \\
And this give us a new Phillips Curve
\[ \boxed{\pi_t = \pi_{t-1} - \omega(u_t - u_N) + \rho_t} \]
It can be written as the accelerationist Phillips Curve
\[ \Delta\pi_t = - \omega(u_t - u_N) + \rho_t  \]

\subsection{Aggregate Supply}
The aggregate supply curve represent the positive relationship between
the \textbf{quantity of output} that businesses are willing to 
produce and the \textbf{inflation rate}.
\smallskip \\
The \textbf{short-run aggregate supply curve (SRAS)} is derived from the 
short-run Phillips curve by replacing the unemployment gap $u - u_N$ with
the output gap $y - y^P$. These two values can be associated empirically using
\textbf{Okun's Law} \\
\includegraphics[width=.5\textwidth, center]
{Screen Shot 2020-10-23 at 3.01.18 PM.png}
which states that
\[ (u_t - u_N) \approx -0.6(y_t - y^P) \]
This substitution gives us the SRAS
\[ \boxed{\pi_t = \pi_{t-1} - \gamma(y_t - y^P) + \rho_t} \]
where $\gamma = -C_{\text{Okun}} \omega$. It indicates how flexible price and 
wage are. In the long-run, as price and wage become completely flexible
$(\gamma \rightarrow \infty)$. This 
gives us the \textbf{long-run aggregate supply curve (LRAS)}
\[ y_t = y^P \]

\section{Monetary Policy}
The Federal Reserve’s dual mandate is \textbf{stable prices} and 
\textbf{maximum employment}, referring to inflation and unemployment. Other
centeral banks in the world have similar goals.
These two objectives can be expressed as a central bank loss function
\[ L = (y_t - y^P)^2 + \beta(\pi_t - \pi^T)^2 \]
To minimize the loss function, we combine the Phillips curve and the IS curve.
This gives us the \textbf{optimal rate rule}.
\[ r_t = r^* + \frac{1}{\zeta_y + \frac{1}{\gamma \beta}}(\pi_t - \pi^T) \]
Central banks use nominal short-term interest rates as their primary
policy tool. In the United States, the FED 

\end{document}