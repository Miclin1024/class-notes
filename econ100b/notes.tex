\documentclass[a4paper]{article}

%% Language and font encodings
\usepackage[english]{babel}
\usepackage[utf8x]{inputenc}
\usepackage[T1]{fontenc}

%% Sets page size and margins
\usepackage[a4paper,top=3cm,bottom=2cm,left=3cm,right=3cm,marginparwidth=1.75cm]{geometry}

%% Useful packages
\usepackage[export]{adjustbox}
\usepackage{amsmath}
\usepackage{graphicx}
\usepackage[colorinlistoftodos]{todonotes}
\usepackage[colorlinks=true, allcolors=blue]{hyperref}
\usepackage{float}
\usepackage{enumerate}
\usepackage{subfig}
\graphicspath{ {./images/} }

\renewcommand{\sb}{\vspace*{5px} \\}

\title{Economic Analysis Macro - Class Notes}
\author{Michael Lin}

\begin{document}
\maketitle

\section{Measurement}

We focus on the major items of macroeconomic interest: \textbf{Economic Activity, Inflation, Unemployment, Interest Rates}

\subsection{National Income Accounting}
\subsubsection*{The Three Approaches to National Income Accounting}
\underline{\textbf{Product Approach}}: the dollar amount of output \textbf{produced}. \sb
The current \textbf{market value} of all \textbf{final goods and services newly produced} in the domestic economy, 
during a specified period of time.
\begin{enumerate}
    \item \textbf{Market Value} \begin{itemize}
        \item Not everything has a market - \textbf{imputed values} must be used. Determined based on \textbf{production cost}. 
        \underline{Example: Fire Department.}
        \item Most non-market goods and services are not included. \underline{Example: Household services.}
        \item Some market goods and services are not included. \underline{Example: Used goods.}
    \end{itemize}
    \item \textbf{Final Goods and Services} \sb
    Goods and services that are \textbf{not completely used up} in the production process.
\end{enumerate}
\underline{\textbf{Expenditure Approach}}: the dollar amount \textbf{spent} by purchasers. \sb
The \textbf{total spending} on \textbf{all final goods and services} produced in the domestic economy during a specified period of time.
\begin{equation}
    \tag{\textbf{The National Income Identity}}
    {\boxed{Y = C + I + G + NX}}
    \end{equation}
    \begin{enumerate}
        \item \textbf{Consumption}\\
        Spending by domestic households on final goods and services.
        \item \textbf{Investment}\\
        Spending for new capital goods\footnote{\textbf{Capital goods} are used to produce other goods and services.
        } (fixed investment) plus inventory investment.
        \item \textbf{Gov. Purchases of Goods and Services} \\
        \textbf{Government Purchases} of goods and services. \sb
        \textbf{Government Transfer Payments} are \underline{\textbf{not}} included in $G$. (Not payments for goods and services) 
        \underline{Example: Medicare, Medicaid, Veterans’ benefits and etc.}
        \item \textbf{Net Exports}: Exports - Imports
    \end{enumerate}
\pagebreak
\textbf{\underline{Income Approach}}: the dollar \textbf{incomes earned} by production. \sb
The \textbf{total income} earned by individuals and businesses in the economy. \sb
\subsubsection*{The Five Income Measures}
\begin{itemize}
    \item \textbf{National Income} = Compensation of Employees + Other Income + Corporate Profits
    \item \textbf{Gross National Product (GNP)} = National Income + Depreciation
    \item \textbf{Gross Domestic Product (GDP)} = GNP + Net Factor Payments
    \item \textbf{Private Disposable Income (PDI)} = GDP + Net Factor Income + Transfer Payments from the Government 
    + Interest Payments on Government Debt - Taxes
    \item \textbf{Net Government Income (NGI)} = Taxes - Government Transfer Payments - Interest Payments on Government Debt
\end{itemize}

\subsection{Inflation}
\subsubsection*{Price Index}
Measures the average level of prices for some specified set of goods and services 
\textbf{relative to the prices in a specified base year.}
\subsubsection*{Three Major Price Indices} \begin{itemize}
    \item \textbf{The Gross Domestic Product (GDP) Deflator:}
    \[ P = 100 * \frac{\text{nominal GDP}}{\text{real GDP}} \]
    \item \textbf{The Personal Consumption Expenditure (PCE) Deflator:}
    \[ P = 100 * \frac{\text{nominal PCE}}{\text{real PCE}} \]
    \item \textbf{The Consumer Price Index (CPI):} \sb
    Measures the average prices of a specified basket of goods and services bought by consumers.
\end{itemize}
\subsubsection*{Measuring Inflation}
Given the price index at time $t$, denoted as $P_t$, the inflation rate $\pi_t$ is expressed as the following
\[ \pi_t \approx \frac{P_t - P_{t-1}}{P_{t-1}} \]
Where with a little bit algebra, we derive the \textbf{Fisher Equation}
\[ i \approx r + \pi \]
$i$ stands for the nominal interest rate, and $r$ is the real interest rate.
\subsubsection*{Unemployment}

\section{Aggregate Production and Productivity}
When labor and capital are separately increased with the other held constant, 
the product increases by \textbf{diminishing increments}.
\subsection{The Cobb-Douglas Production Function}
Their assumptions for $Y(K, L)$: \begin{itemize}
    \item Output $Y$ is a function of capital $K$ and labor $L$
    \item Constant returns to scale: \sb
    If $K - m * K$ and $L - m * L$ then $Y - m * Y$
\end{itemize}
Their proposed solution:\begin{equation}
    \tag{\textbf{The Cobb-Douglas Production Function}}
    \boxed{Y = AK^\alpha L^{1 - \alpha}}
\end{equation}
Shows how much output can be produced from givenamounts of 
capital and labor with a given level of total-factor productivity $A$ 
(also referred to as "technology"). \sb
Taking logs of both sides and subtract $\ln(L)$
\[\ln(\frac{Y}{L}) = \ln{A} + \alpha \ln(\frac{K}{L})\]
Notice that we now have a linear equation that we can analyze
\begin{figure}[H]%
    \centering
    \subfloat[\centering Linear Approx]{{\includegraphics[width=6cm]{agg_production_linear.png} }}%
    \qquad
    \subfloat[\centering Diminishing Marginal Return to K]{{\includegraphics[width=7cm]{Screen Shot 2020-09-18 at 4.12.59 PM.png} }}%
    \caption{}%
    \label{fig:1}%
\end{figure}

The production function can be drawn as either \begin{itemize}
    \item Output as a function of capital (As shown in Figure 1(b))
    \item Output as a function of labor
\end{itemize}
\begin{equation}
    \tag{Marginal Product of Capital}
    \begin{split}
        \text{MPK} & \equiv \frac{\partial Y}{\partial K} \\
        & = \alpha A (\frac{L}{K})^{1 - \alpha}
    \end{split}
\end{equation}

\subsection{Understanding Shocks}
Consider the Cobb-Douglas Production Function
\[ Y = AK^\alpha L^{1 - \alpha} \]
When $AL^{1 - \alpha}$ goes down, output goes down across all input $K$
\begin{figure}[H]%
    \centering
    \subfloat[\centering Reduced Y Across K]{{\includegraphics[width=6cm]{Screen Shot 2020-09-18 at 4.48.01 PM.png} }}%
    \qquad
    \subfloat[\centering Reduced MPC]{{\includegraphics[width=6cm]{Screen Shot 2020-09-18 at 4.48.27 PM.png} }}%
    \caption{Shock in $AL^{1 - \alpha}$}%
    \label{fig:Shock}%
\end{figure} \ \\
Moveover, we can see that the slope of the post-shock production curve is smaller 
than the pre-shock curve, which indicates the decrease of \textit{MPK}. As shown in 
Figure 2(b), this also indicates the decrease of \textbf{real rental cost of capital}. \sb
Similarily, when $AK^{\alpha}$ goes down, we will have a lower output across all input of capital.

\subsection{Full-Employment Output}
Full-employment, or potential, outputis the level of outputwhen the labor market is in the \textbf{long-run equilibrium}.
\[ Y^P = AK^\alpha L_S^{1-\alpha} \]

\section{The Solow-Swan Model: Long-Run Growth}
Over the decades, we discovered that the growth of GDP is roughly \textbf{linear} in the \textbf{logrithmic domain}. 
\[ \ln(Y(t)) = c + g_Y t \]
Following this idea, we can get a model for describing the output at time $t$
\[ Y(t) = Y(0)e^{g_Y t} \]
\includegraphics[width=0.5\textwidth, center]{Screen Shot 2020-09-25 at 5.36.52 PM.png}
We can see that our log-linear fit is pretty reasonable. The fluctuations about the trend are referred to as \textbf{bussiness cycles}.

\subsection{Where Does Growth Come From?}
Let's begin with the Cobb-Douglas Production Function
\[ Y = AK^\alpha L^{1 - \alpha} \]
Taking the log of both sides, then take the derivative
\begin{equation}
    \tag{Growth Accounting Formula}
    \boxed{\frac{1}{Y}\frac{dY}{dt} = \frac{1}{A}\frac{dA}{dt} + \alpha \frac{1}{K}\frac{dK}{dt} + (1-\alpha)\frac{1}{L}\frac{dL}{dt}}
\end{equation}
Approximate the result with $\Delta t = 1$, we have
\[ \frac{\Delta Y}{Y} = \frac{\Delta A}{A} + \alpha \frac{\Delta K}{K} + (1 - \alpha)\frac{\Delta L}{L} \]
Sometimes it's also written as $g_Y = g_A + \alpha g_K + (1 - \alpha)g_L$ \sb
For example, in the United States where $\alpha \approx 0.3$, we have
\[g_Y = g_A + 0.3 g_K + 0.7 g_L\]
This tells us how the growth of one or more of the variables in the production function attribute to the overall increase in output.\sb
Research has shown that \textbf{productivity growth} is a more important source of variation in growth rates across countries than is growth 
in capital or labor(a.k.a. factor accumulation).

\subsection{The Solow-Swan Model}
The Solow-Swan Model was developed in 1950s to determine \textbf{capital accumulation}, which affects \textbf{economic growth}. \sb
To derive this, we assume that a \textbf{constant fraction $s$ of output} $Y$ is saved, and the capital \textbf{depreciates at a constant rate} $\delta$.
\begin{equation}
    \tag{Capital Accumulation Equation}
    \frac{dK}{dt} = sAK^{\alpha}L^{1-\alpha} - \delta K
\end{equation}
The goal is to solve this differential equation so we get $K$ as a function of $t$ \sb
We make the following assumptions: \begin{itemize}
    \item Replace total factor of productivity $A$ with \textbf{labor efficiency} $E$, and assume that $E(t)$ has a \textbf{log-linear growth rate}
    \begin{align*}
        E(t) &= A(t)^{1/(1-\alpha)} \\
        E(t) &= E(0)e^{g_E t}
    \end{align*}
    \item Assume that $L$ also grows in log-linear rate.
    \[ L(t) = L(0)e^{g_L t} \]
    \item Instead of examine the aggregate capital $K$, we look at the the \textbf{capital per worker per unit of worker efficiency}, denoted by $\kappa$. This 
    is also referred to as \textbf{normalized capital}.
    \[ \kappa(t) = \frac{K(t)}{E(t)L(t)} \]
\end{itemize}
Plug in our expression for $\kappa(t)$, $L(t)$, and $E(t)$, we have the capital accumulation equation for $\kappa$
\[ \frac{d\kappa}{dt} = s\kappa^\alpha - (g_E + g_L + \delta)\kappa \]
The solution $\kappa(t)$ is important because it is directly related to \textbf{per-capita income} by
\[ \frac{Y(t)}{L(t)} = \kappa(t)^\alpha E(t) \]
% 
Generally we have two solutions of interest
\begin{itemize}
    \item The steady-state solution $\kappa^*$
    \[ \frac{d\kappa}{dt} = 0 \]
    \item The general solution $\kappa(t)$
\end{itemize}


\subsection{The Solutions}

\subsubsection*{Stead-state Solution}
Follows from the capital accumulation equation
\[ \frac{d\kappa}{dt} = s\kappa^\alpha - (g_E + g_L + \delta)\kappa = 0 \]
so \begin{align*}
    s\kappa^\alpha &= (g_E + g_L + \delta)\kappa \\
    \kappa^{1-\alpha} &= \frac{s}{g_E + g_L + \delta}
\end{align*}
and
\begin{equation*}
    \boxed{\kappa^* = (\frac{s}{g_E + g_L + \delta})^\frac{1}{1 - \alpha}}
\end{equation*}
In the steady state, per-capita income is given by
\[ \frac{Y(t)}{L(t)}  = \kappa^*(t)^\alpha E(t) = (\frac{s}{g_E + g_L + \delta})^\frac{1}{1 - \alpha}E(t) \]
and from which follows that
\[ g_{Y/L} = g_E \]
In the steady state $\kappa(t) = \kappa^*$ and is not changing, so
\[ \kappa(t) = \frac{K(0)e^{g_Kt}}{E(0)e^{g_Kt}L(0)e^{g_Kt}} 
= \frac{K(0)}{E(0)L(0)}e^{(g_K-g_E-g_L)t} = \kappa^* \]
Therefore $g_{Y/L} = g_E = g_K - g_L$

\subsubsection*{General Solution}
The general solution can be written in 'gap' form
\[ \text{gap}(t) = \text{gap}(0)e^{-t/\tau} \]
where
\[ \text{gap}(t) = \kappa(t)^{1-\alpha} - \frac{s}{g_E + g_L + \delta} \]
Overtime $\text{gap}(t)$ converges to 0, with the \textbf{half life} given by
\[ \boxed{t_{1/2} = \frac{\ln 2}{(1 - \alpha)(g_E + g_L + \delta)}} \]

\subsection{Shock Analysis}
Four important components\begin{itemize}
    \item $\kappa^*(t)^\alpha E(t)$: \textbf{The balanced growth} 
    \[\kappa^* = (\frac{s}{g_E + g_L + \delta})^\frac{1}{1 - \alpha}\]
    \item $Y(t)/L(t)$: \textbf{Per capita income}
    \[ \frac{Y(t)}{L(t)} = \kappa(t)^\alpha E(t) = (\frac{K(t)}{E(t)L(t)})^\alpha E(t) \]
    \item $t_{1/2}$: \textbf{Half life} -- How fast per capita income converges to the 
    balanced growth after a shock
    \[ t_{1/2} = \frac{\ln 2}{(1 - \alpha)(g_E + g_L + \delta)} \]
    \item $g_{Y/L}$: \textbf{Long term growth rate}
    \[ g_{Y/L} = g_E \]
\end{itemize}

\subsection{Endogenous Growth Theory - the Romer Model}
If we are looking for the factors that help the economy grows faster, we soon 
realize that factors such as the saving rate, labor-force growth rate can rise 
and fall forever, but efficiency can always improve. \sb
In endogenous growth theory, we look at \textbf{technology as a 
production input.} It differs from capital and labor in two important 
characteristics: \begin{itemize}
    \item \textbf{Non-rivalry: } More than one person can use the factor 
    at any given time.
    \item \textbf{Non-excludability: } One person cannot prevento thers 
    from using the factor.
\end{itemize}
Endogenous growth theory is an attempt to explain how 
and why technology can increase
endogenously and, thereby sustained increases in income-per-worker. 
It is often referred to as the \textbf{Romer Model}.
\pagebreak

\subsubsection*{The Romer Model}
In the Romer Model, \textbf{labor} is allocated to the production of \begin{itemize}
    \item \textbf{Goods and services} $L_P$
    \item \textbf{New technology} $L_E$
\end{itemize}
The total labor supply $L  = L_P + L_E$ is assumed to be fixed. Let $\gamma$ 
denotes the ratio of total labor participating in R\&D.
\[ L_E = \gamma L \]
The production function for technology
\[ \frac{dE}{dt} = \chi E L_E \]
where $\chi$ indicates how productive labor is in producing ideas. \sb
%
If we take the technology production function and divide both side by $E$
\[ \frac{1}{E}\frac{dE}{dt} \equiv g_E = \chi L_E \]
Substituting $L_E$, we get the \textbf{Romer Model}
\[ \boxed{g_E = \chi \gamma L} \]

\section{Bussiness Cycles}

\section{IS Curve}
From the production function we can derive the economy's 
\textbf{full-employment output level} 
(how much the economy could produce). This is the basis for the 
economy's \textbf{aggregate supply.} \sb
%
Now the focus shifts to how much is the demand in the economy, a.k.a the 
\textbf{aggregate demand}. \sb
%
Aggregate demand and aggregate supply are crucial for explaining \textbf{short-run 
fluctuations} in economic activity.
\subsection{Planned Expenditures}
There are two types of expenditure \begin{itemize}
    \item \textbf{Acutal Expenditures} $Y$: Total amount of spending on 
    domestically produced goods and services that households, businesses, the 
    government, and foreigners acutally make. \textbf{This is equivalent to the \underline{total 
    output} actually produced in the economy.}
    \item \textbf{Planned Expenditures} $T^{pl}$: Total amount of spending on 
    domestically produced goods and serves that households, businesses, the 
    government, and foreigners want to make. \textbf{This is quivalent to the 
    \underline{aggregate demand}.}
\end{itemize}
The economy is in \textbf{goods market equilibrium} when actual expenditures equal planned 
expenditures
\[ Y = Y^{pl} \]
Similarily to the national income identity, $Y^{pl}$ can be expressed as
\[ Y^{pl} = C + I^{pl} + G + NX \]

\subsubsection*{The Consumption Function}
Consumption expenditures are represented by the \textbf{consumption function}
\[ C = \overline{C} + \text{mpc} \cdot Y^d - \zeta_c  r_L \]
where
\begin{align*}
    \overline{C} &\equiv \text{Autonomous Consumption Expenditure} \\
    Y^d &\equiv \text{Disposable Income} = Y - T \\
    \text{mpc} &\equiv \text{Marginal Propensity to Consume} \\
    \zeta_c &\equiv \text{Sensitivity of $C$ to $r_L$} \\
    r_L &\equiv \text{Real Lending Interest Rate}
\end{align*}

\end{document}