\documentclass[a4paper]{article}

%% Language and font encodings
\usepackage[english]{babel}
\usepackage[utf8x]{inputenc}
\usepackage[T1]{fontenc}

%% Sets page size and margins
\usepackage[a4paper,top=3cm,bottom=2cm,left=3cm,right=3cm,marginparwidth=1.75cm]{geometry}

%% Useful packages
\usepackage[export]{adjustbox}
\usepackage{amsmath}
\usepackage{graphicx}
\usepackage[colorinlistoftodos]{todonotes}
\usepackage[colorlinks=true, allcolors=blue]{hyperref}
\usepackage{float}
\usepackage{enumerate}
\usepackage{subfig}
\usepackage{algorithm}
\usepackage{amsthm}
\usepackage[noend]{algpseudocode}
\graphicspath{ {./images/} }

\renewcommand{\sb}{\vspace*{4px} \\}
\renewcommand{\arraystretch}{1.5}

\title{CS 170 - Class Notes}
\author{Michael Lin}

\begin{document}
\maketitle

\section{Divide and Conquer}
\subsection{Master Theorem}
For a recurrence relation that takes the form:
\[ T(n) = aT(\frac{n}{b}) + O(n^d) \]
then
\[ T(n) = \begin{cases}
    \Theta(n^d) & \text{ if } d > \log_b a \ \  \text{The algorithm is root heavy} \\
    \Theta(n^d \log n) & \text{ if } d = \log_b a  \\
    \Theta(n^{\log n}) & \text{ if } d < \log_b a \ \  \text{The algorithm is leaf heavy}
\end{cases} \]
\subsection{Examples}
\subsubsection{Median Finding}
\subsubsection{Matrix Multiplication}

\section{Fast Fourier Transform}

First, we know that evaluating a polynomial $A(x)$ takes only linear time because we as soon as we compute $x^n$ we can multiply by $x$ and get $x^{n+1}$
\begin{equation}
    \tag{Horner's Rule}
    A(x) = a_0 + x(a_1 + x(a_2 + \dots x(a_{n-1})))
\end{equation}
Moreover, addition also take linear time because for $C(x) = A(x) + B(x)$, we have $c_k = a_k + b_k$. However for matrix multiplication, we would have to match 
each term in one polynomial to every term in the other, resulting in an $O(n^2)$ runtime.
\[ c_k = \sum_{j=0}^k a_j b_{k-j} \]
FFT can be used to achieve multiplication in just $O(n \log n)$ time.

\pagebreak
\subsection{Representation of Polynomial}
\begin{itemize}
    \item \underline{\textbf{Coefficient Vector:}} $a_0 + a_1 x + a_2 x^2 \dots$
    \item \textbf{\underline{Roots Representation:}} $c(x-r_0)(x-r_1) \dots (x-r_{n-1 })$
    \item \textbf{\underline{Samples Representation:}} $(x_k, y_k) \ \text{for} \ k = 0, 1 \dots n-1$ where $x_k$'s distinct
\end{itemize}
We can convince ourselves that the runtime of the three operations for the representation can be summerized as followed:
\begin{center}
    \begin{tabular}{ ||c|c|c|c|| }
        \hline
        Operations & Coeffs & Roots & Samples \\
        \hline \hline
        Evaluation & $O(n)$ & $O(n)$ & $O(n^2)$ \\
        \hline
        Addition & $O(n)$ & $\infty$ & $O(n)$ \\
        \hline
        Multiplication & $O(n^2)$ & $O(n)$ & $O(n)$ \\
        \hline
    \end{tabular}
\end{center}
Ultimately, FFT is a algorithm that can convert a polynomial between its coefficient and samples representations in $n \log n$ time

\subsection{The Algorithm - Divide and Conquer}
\textbf{Goal:} $A(x)$ for $x \in X$ -- From coefficients view to samples view \sb
\textbf{Divide:} Divide into even and odd coefficient
\[ A_{\text{even}}(x) = \sum_{k=0}^{n/2}a_{2k}x^k \]
\[ A_{\text{odd}}(x) = \sum_{k=0}^{n/2}a_{2k+1}x^k \]
\textbf{Combine:} Obtain $A(x)$ given $A_{\text{even}}(x)$ and $A_{\text{odd}}(x)$
\[ A(x) = A_{\text{even}}(x^2) + xA_{\text{odd}}(x^2) \]
\textbf{Conquer:} Therefore, we can recursively compute
\[ A_{\text{even}}(y) \text{ and } A_{\text{odd}}(y) \text{ for } y \in X^2 = \{x^2 | x \in X\} \]
Until we reach $n = 1$, that is when we have polynomial of degree one. \sb
The big picture here is that in each recursive step, we compute a different polynomial of half the degree, 
but with a different set of the same size ( $X^2$ has the same size as $X$ ). Here we obtain the recurrence
\[ T(n, |X|) = 2 T(n/2, X) + O(n + |X|) \]
This is not solvable by the Master's Theorem as it involves two variables. However, by drawing the recursion tree we soon discover that 
this is a bad recurrence -- $|X|$ starts at $N$ and never goes down. \sb
\includegraphics[width=0.4\textwidth, center]{IMG_C1B43842E2E5-1.jpeg}
At the bottom of the tree there are $2^{\log n}$ leaves, which would still give us $n^2$ running time. \sb
Nevertheless, this gives us an idea how to do better. If in each recursive step, we manage to also reduce the size that we are sampling the 
polynomial at. Or in another word, choose a set of $X$ such that $ |X^2| = |X| / 2$. We can achieve $n \log n$ running time.

\subsection{Nth Roots of Unity}
Let's start with $X = \{1\}$ when $|X|=1$. After that, what if we want two values in $X$? 
Remember that we want the feature where we square the 
values, we only have one value. Naturally, we will think of $X = \{-1, 1\}$. 
Moreover, we realize that there are two square roots for every number. 
We can keep taking square roots for all of them, and when we square them, it collapses into half of the size.
\begin{align*}
    |X|=1 \ \Rightarrow \ &X=\{1\} \\
    |X|=2 \ \Rightarrow \ &X=\{-1, 1\} \\
    |X|=4 \ \Rightarrow \ &X=\{i, -i, -1, 1\} \\
    |X|=8 \ \Rightarrow \ &X=\{\pm \frac{\sqrt{2}}{2}(1+i), \pm \frac{\sqrt{2}}{2}(1-i) , i, -i, -1, 1\} \\
    \dots
\end{align*}
If we were to draw the numbers in the same set on a plane, you will see that they are evenly distributed in a unit circle. \\
\includegraphics[width=0.4\textwidth, center]{IMG_93A6F1D0EE96-1.jpeg}
It is because of such geometric property that we are able to reduce the size of the set in half when we square all of its elements. When you 
square a number, it's essentially doubling its angle, so half of its elements will overlap with each other. \sb
We call the elements that satisfies $x^n = 1$ the \textbf{nth roots of unity}, and an element is said to be \textbf{primitive} if n is 
the smallest number that satisfies the condition, denoted by $\omega_n$.\sb
We can also express the nth roots of unity using Euler's formula. Since the points are evenly spaced, we have
\[ \omega_n = e^{i\tau / n} \text{  where  } \tau = 2\pi \]
So if we go back to 2.2, and instead of chosing an arbitrary set of $X$, we sample the 
polynomial at \{1, $\omega_n, \omega_n^2, \dots \omega_n^{n-1}$\}, we have a divide and conquer algorithm that 
runs in $n \log n$ time. That algorithm is called \textbf{Fast Fourier Transform (FFT)} \sb
Furthermore, we can re-express the evaluation of the polynomial $a_0 + a_1x + a_2x^2 + \dots + a_{n-1}x^{n-1}$ at 
$x = \{1, \omega, \omega^2, \dots \omega^{n-1}\}$ using the following matrix multiplication \sb
\includegraphics*[width=0.6\textwidth, center]{Screen Shot 2020-09-24 at 2.05.08 PM.png}
The corresponding transformation (denoted as $F_n$) is also known as the \textbf{Discrete Fourier Transform}. We notice that $F_n = F_n^T$ and 
$(F_n)_{jk} = \omega^{jk} = e^{i(jk)\tau/n}$. All the entries of $F_n$ are $n$th roots of unity, 
which means raising any of them to the $n$th power gives 1. 

\subsection{Polynomial Multiplication}
Now that we have our FFT, doing polynomial multiplication becomes quite simple. For polynomial $A$ and $B$, first thing we need is to compute 
their FFT
\[ A^* = \text{FFT}(A) \text{  and  } B^* = \text{FFT}(B) \]
What it does is that it converts $A$ and $B$ from coefficient vector into sample vector. Essentially, we have the samples of $A$ and $B$, at 
the nth roots of unity $x_k = e^{ik\tau/n}$. Therefore the transformed version of our desired polynomial $C^*$ is just the result 
of element-wise multiplication of $A^*$ and $B^*$
\[ C_k^* = A_k^* B_k^* \]
Then to compute C, we need the inverse fast fourier transform of $C^*$
\[ C = F_n^{-1}(C^*) \]
Luckily, this is not too hard. Because $F_n$ is almost unitary (the complex equivalent of orthonormal) except that 
each column has a length of $\frac{n}{2}$ instead of 1, and a very nice property of a unitary matrix is that 
the inverse is just its conjugate transpose. Here, we can divide each column in both
$F_n^H$ and $F_n$ by $\frac{n}{2}$ to get a matrix whose columns are orthonormal, which gives us the following identity
\[ \frac{1}{n}F_n^H F_n = I \]
Furthermore, we notice that $F_n$ is symmetric, which means $F_n^H = \overline{F_n}$
\[ F_n^{-1} = \frac{1}{n} \overline{F_n} \]
Therefore, for polynomial multiplication, we will first transform them into sample representation using FFT in $n \log n$ time, then
do the multiplication in linear time, and use IFFT to transform back to coefficient representation in $n \log n$ time. The result 
is a polynomial multiplication algorithm that runs in $n \log n$ time.

\section{Graph Search}
Recall that a graph is defined as $G = (V, E)$, where $V = \text{Set of vertices, }
E = \text{Set of edges}$. The elements of $E$ may be \textbf{undirected pairs}, denoted by 
$e = \{v, w\}$, or \textbf{directed pairs}, denoted by $e = (v, w)$. \sb
The applications of graph search includes: web crawling, social networking, network 
broadcast, garbage collection, solving puzzles and games.

\subsection{Case Study: Pocket Cube}
Suppose we have a 2x2x2 rubik's cube. We can model it using a configuration graph.\sb
\textbf{\underline{Configuration Graph:}} \begin{itemize}
    \item \textbf{Vertex for each possible state of the cube.} For a 2x2x2 cube, 
    because we have 8 cubies, an each cubies has 3 possible twist 
    \#vertices = $8! \cdot 3^8 = 264,539,520$.
    \item \textbf{Edge for each possible move.} Since every move is "undoable", 
    the graph is undirected.
\end{itemize}
Amongst all the vertices, there's a certain \textbf{solved states}, and it is connected 
to all the possible moves from there. The graph will keep extending, until it 
exhausts all of the states. The maximum depth of this graph is also known as 
the \textbf{diameter} of the graph, which represents the maximum number of steps 
it takes to solve the rubik's cube. In the case of a 2x2x2, this number is 11. 
For 3x3x3, the number is 20, and it's roughly $\Theta(n^2 / \lg n)$ for NxNxN.

\subsection{Graph Representation}

\subsubsection*{\underline{Adjacency List}} For each vertex $u \in V$, we have 
$\text{adj}[u]$ that stores all of the neighbors of $u$. $\text{adj}[u]$ could 
be a linked list, or a hash set based on the hashed address of the vertices. 
More formally, $\text{adj}[u] = \{ v \in V | (u, v) \in E \}$ \sb
In an object-oriented scenario, it is possible to defined \textbf{u.neighbors = adj[u]}. 
However, despite that this representation is cleaner, 
it means that the vertex $v$ can only be involved in one graph structure, whereas 
if we are using adjacency lists, we can define multiple graphs using the same 
vertices.

\subsubsection*{\underline{Implicit Representation}} In some cases such as the 
rubik's cube, we don't want to build the whole space of $\text{adj}[u]$. Instead 
we would have $\text{adj}(u)$ being a function, or u.neighbors() being a method. 
\textbf{Compared with adjacency list this uses zero space because we are computing the 
neighbors on the go}, and if we are lucky we don't need to build the whole graph, 
we just need enough of it for us to find the answer. \sb
For our rubik's cube problem, computing all the possible states for a larger cube 
would not be possible. But representing a state is easy, and knowing what's 
reachable from that state is also relatively easy. 

\subsection{Breadth First Search}

\subsubsection*{Goal} \begin{itemize}
    \item Visit all nodes reachable from a given node $u \in V$
    \item Achieve $O(V + E)$
    \item Look at nodes reachable in \{ 0, 1, 2 \dots \} moves
    \item Avoid duplicates
\end{itemize}
%
\begin{algorithm}
    \caption{Breadth First Search} \begin{algorithmic}[1] 
        \Procedure{BFS}{$s, adj$} \Comment{Given the starting 
        vertex $s$ and the graph as an adjacency list}
            \State level = \{ s: 0 \} \Comment{The 
            minimum steps to reach a vertex $u$}
            \State parent = \{ s: None \} \\ \Comment{Optional, 
            the parent vertex of $u$. This later helps us 
            find the \textbf{shortest path}}
            \State i = 1
            \State frontier = [ s ] \Comment{All vertices at 
            level $i-1$}
            \While{frontier} \Comment{frontier is not empty}
                \State next = [ ]
                \For{$u$ \textbf{in} frontier}
                    \For{v \textbf{in} adj[u]} \Comment{Look at all 
                    vertices at the next level}
                        \If{v \textbf{not in} level} \\ \Comment{\textbf{Checking 
                        duplicates} -- if we have visited v, it should have a level}
                            \State level[$v$] = i
                            \State parent[$v$] = $u$
                            \State next.append($v$)
                        \EndIf
                    \EndFor
                \EndFor
                \State frontier = next
                \State $i = i + 1$
            \EndWhile
        \EndProcedure
    \end{algorithmic}
\end{algorithm}
%
\pagebreak
\subsubsection*{Runtime Analysis}
When we look at frontier we see that the edges only enter it once, 
because before one vertex enters frontier it would be marked in 
level. Therefore in a single BFS procedure with starting point 
$s$ and adjacency list $adj$ the running time of would be the 
sum of all the vertices in the adjacency list.
\[ \sum_{u \in V}|\text{adj}[u]| = \begin{cases}
    2|E| & \text{undirected graph} \\
    |E| & \text{directed graph}
\end{cases} \]
On top of that, we also spend order V time because we need to 
touch every vertex. So the total running time would be 
$\Theta(V + E)$

\subsection{Depth First Search}
Similar to BFS, DFS can be used to explore the vertices in a graph. However, 
unlike BFS which explore a graph layer by layer, DFS uses recursion which gives 
a more straightforward algorithm for the \textbf{connectivity} of a graph.
%
\begin{algorithm}
    \caption{Depth First Search} \begin{algorithmic}
        \Procedure{DFS-Visit}{$s, adj$} \Comment{Recursivly visit all the 
        nodes reachable from $s$}
            \For{v \textbf{in} adj[s]}
                \If{v \textbf{not in} parents}
                    \State parents[v] = s
                    \State \Call{DEF-Visit}{v, adj}
                \EndIf
            \EndFor
        \EndProcedure
\\
        \Procedure{DFS}{$V, adj$}
            \State parents = \{ \}  \\ \Comment{A list keeping track of 
            all visited nodes also works}
            \For{s \textbf{in} V}
                \If{s \textbf{not in} parents} \Comment{This step is to find all
                 possible starting points}
                    \State parents[s] = None
                    \State \Call{DFS-Visit}{s, adj} \\ \Comment{If a node has 
                    not yet been visited, we recursivly explore all 
                    connected nodes from $s$}
                \EndIf
            \EndFor
        \EndProcedure
    \end{algorithmic}
\end{algorithm}
%
\subsubsection*{Runtime Analysis} 
In procedure DFS, we visit each vertex once so that's $O(V)$ to begin with, 
and inside each iteration of vertex $v$, we pay for what's inside the 
adjacency list $\text{adj}[v]$. So just as in the case for BFS, we need to add 
order E time, and the result would be $\Theta(V + E)$.
%
\subsubsection*{Edge Classification}
The edges we traverse as we execute a depth-first search can be classified 
into four edge types. \begin{enumerate}[1.]
    \item If $v$ is visited for the first time as we traverse the edge 
    ($u$, $v$), then the edge is a \textbf{tree edge}.
    \item Else, $v$ has already been visited: \begin{enumerate}
        \item If $v$ is an ancestor of $u$, then edge ($u$, $v$) is a \textbf{back edge}.
        \item Else, if $v$ is a descendant of $u$, then edge ($u$, $v$) is a \textbf{forward edge}.
        \item Else, if $v$ is neither an ancestor or descendant of $u$, then edge ($u$, $v$) is a \textbf{cross edge.}
    \end{enumerate}
\end{enumerate}
\includegraphics*[width=0.8\textwidth, center]{Screen Shot 2020-09-27 at 8.49.21 PM.png}
This is useful for many problems, here we are going to explore two of 
them: \textbf{Cycle Detection} and \textbf{Topological Sort}.

\subsubsection*{Cycle Detection}
We claim that G has a cycle $\Leftrightarrow$ DFS(G) has a back edge.
\begin{proof}
    We need to proof the proposition in both direction. \sb
    $\Leftarrow$: If there's a back edge, by definition, it makes a cycle. \sb
    $\Rightarrow$: Given a cycle $\{v_0, v_1, \dots, v_k\}$, we can assume that 
    $v_0$ is the first vertex in the cycle we visited during DFS. By induction, 
    we can prove that we will visit $v_k$ before finish $v_0$. This argument can 
    be made for any $v_i$, we can think of it as "balanced parentheses"
    $\underset{u}{(} \dots \underset{k}{(} \dots \underset{k}{)} \dots \underset{u}{)}$
    \sb 
    Therefore, when we visited $v_k$, $v_0$ will still be on the stack. We will 
    consider ($v_k$, $v_0$), and because $v_0$ is still being processed, we will 
    mark it as a back edge.
\end{proof}

\subsubsection*{Topological Sort}
\textbf{\underline{Example:}} Given Directed Acylic Graph (DAG), where vertices represent tasks \& edges 
represent dependencies, order tasks without violating dependencies. \sb
The topological sort result 
is the \textbf{reverse of DFS finishing times} (time at which DFS-Visit($v$) finishes).
\begin{proof}
    To prove the correctness, we need to prove that for any edge $e=(u,v)$ 
    $u$ is ordered before $v$, that is, in DFS-Visit, 
    $v$ finishes before $u$ finishes. \begin{itemize}
        \item if $u$ visited before $v$: \sb
        Before visit to $u$ finishes, will visit $v$ -- via ($u$, $v$) or otherwise. 
        By "balanced parentheses", $v$ will finish before $u$.
        \item if $v$ visited before $u$: \sb
        Because graph is acyclic, \textbf{$u$ cannot be reached from $v$}. That means 
        visit to $v$ finishes before visiting $u$.
    \end{itemize}
\end{proof}

\subsection{Connected Components}

For an undirected graph $G$, BFS($v$) or DFS($v$) will visit every other vertex 
in the same connected component as $v$. We can mark every vertex visited from a 
BFS/DFS from $v$ as being “owned” by $v$. As we iterate through all the 
vertices, we execute a 
BFS/DFS starting from a vertex if it has no owner (i.e. it is part of an 
undiscovered connected component) and mark all the vertices visited in that 
BFS/DFS.\sb
%
The runtime of this algorithm is $\Theta(V + E)$ since each vertex is visited 
twice (once by iterating through it in the outer loop, another by visiting it 
in BFS/DFS) and each edge is visited once (in BFS/DFS).\sb
However, the algorithm does not work with directed graph. For undirected graphs, 
finding a path from u to v implies that there exists a path from v to u. This 
is not the case for directed graphs. 

\subsubsection*{Strongly Connected Components}
We define Strongly Connected Components ( ) as components in directed graphs where any two vertices has a 
path in between each other. \sb
%
The intuition that will help us separate a directed graph into strongly 
connected components is realizing that \textbf{a SCC with its edges’ directions 
reversed is still a SCC.} We will introduce $G^T$, which is the transpose 
of directed graph $G$. $G^T$ and $G$ are the same graph except the edge 
directions are reversed in $G^T$ , i.e. if edge ($u$, $v$) is in $G$, then 
the edge $(v,u)$ is in $G^T$.


\section{Shortest Path in Weighted Graphs}
\underline{\textbf{Definition:}}
\[ G(V, E, W) \]

\section{Dynamic Prgramming} 
Dynamic Programming (DP) is a powerful, general algorithmic design technique. 
It can be think of as "careful brute force", or "subproblems + reuse". Similar 
to divide and conquer, you take a problem, split it into subproblems, solve the 
subproblems and reuse the solution to the subproblems.
\subsection{Fibonacci Number}
\[ F_1 = 1 \text{ , } F_2 = 1 \]
\[ F_n = F_{n-1} + F_{n-2} \]
\underline{\textbf{Goal:}} Compute $F_n$ \sb
Here, the naive approach would be just recursively calling $F_n = F_{n-1} + F_{n-2}$ 
until we reach $n=2$. However this give us the recurrence
\[ T(n) = T(n-1) + T(n-2) + \Theta(1) \]
% \begin{algorithm}
%     \caption{Naive Fibonacci Algorithm} \begin{algorithmic}
%         \Procedure{FibNaive}{$n$}
%         \If{$n \le 2$} f = 1
%         \Else \State f = \Call{FibNaive}{$n-1$} + \Call{FibNaive}{$n-2$}
%         \EndIf
        
%         \EndProcedure
%     \end{algorithmic}
% \end{algorithm}

\end{document}